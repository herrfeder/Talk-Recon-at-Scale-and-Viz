\PassOptionsToPackage{unicode=true}{hyperref} % options for packages loaded elsewhere
\PassOptionsToPackage{hyphens}{url}
%
\documentclass[ignorenonframetext,]{beamer}
\usepackage{pgfpages}
\setbeamertemplate{caption}[numbered]
\setbeamertemplate{caption label separator}{: }
\setbeamercolor{caption name}{fg=normal text.fg}
\beamertemplatenavigationsymbolsempty
% Prevent slide breaks in the middle of a paragraph:
\widowpenalties 1 10000
\raggedbottom
\setbeamertemplate{part page}{
\centering
\begin{beamercolorbox}[sep=16pt,center]{part title}
  \usebeamerfont{part title}\insertpart\par
\end{beamercolorbox}
}
\setbeamertemplate{section page}{
\centering
\begin{beamercolorbox}[sep=12pt,center]{part title}
  \usebeamerfont{section title}\insertsection\par
\end{beamercolorbox}
}
\setbeamertemplate{subsection page}{
\centering
\begin{beamercolorbox}[sep=8pt,center]{part title}
  \usebeamerfont{subsection title}\insertsubsection\par
\end{beamercolorbox}
}
\AtBeginPart{
  \frame{\partpage}
}
\AtBeginSection{
  \ifbibliography
  \else
    \frame{\sectionpage}
  \fi
}
\AtBeginSubsection{
  \frame{\subsectionpage}
}
\usepackage{lmodern}
\usepackage{amssymb,amsmath}
\usepackage{ifxetex,ifluatex}
\usepackage{fixltx2e} % provides \textsubscript
\ifnum 0\ifxetex 1\fi\ifluatex 1\fi=0 % if pdftex
  \usepackage[T1]{fontenc}
  \usepackage[utf8]{inputenc}
  \usepackage{textcomp} % provides euro and other symbols
\else % if luatex or xelatex
  \usepackage{unicode-math}
  \defaultfontfeatures{Ligatures=TeX,Scale=MatchLowercase}
\fi
% use upquote if available, for straight quotes in verbatim environments
\IfFileExists{upquote.sty}{\usepackage{upquote}}{}
% use microtype if available
\IfFileExists{microtype.sty}{%
\usepackage[]{microtype}
\UseMicrotypeSet[protrusion]{basicmath} % disable protrusion for tt fonts
}{}
\IfFileExists{parskip.sty}{%
\usepackage{parskip}
}{% else
\setlength{\parindent}{0pt}
\setlength{\parskip}{6pt plus 2pt minus 1pt}
}
\usepackage{hyperref}
\hypersetup{
            pdftitle={Example Presentation},
            pdfauthor={David Lassig},
            pdfborder={0 0 0},
            breaklinks=true}
\urlstyle{same}  % don't use monospace font for urls
\newif\ifbibliography
\usepackage{color}
\usepackage{fancyvrb}
\newcommand{\VerbBar}{|}
\newcommand{\VERB}{\Verb[commandchars=\\\{\}]}
\DefineVerbatimEnvironment{Highlighting}{Verbatim}{commandchars=\\\{\}}
% Add ',fontsize=\small' for more characters per line
\newenvironment{Shaded}{}{}
\newcommand{\AlertTok}[1]{\textcolor[rgb]{1.00,0.00,0.00}{\textbf{#1}}}
\newcommand{\AnnotationTok}[1]{\textcolor[rgb]{0.38,0.63,0.69}{\textbf{\textit{#1}}}}
\newcommand{\AttributeTok}[1]{\textcolor[rgb]{0.49,0.56,0.16}{#1}}
\newcommand{\BaseNTok}[1]{\textcolor[rgb]{0.25,0.63,0.44}{#1}}
\newcommand{\BuiltInTok}[1]{#1}
\newcommand{\CharTok}[1]{\textcolor[rgb]{0.25,0.44,0.63}{#1}}
\newcommand{\CommentTok}[1]{\textcolor[rgb]{0.38,0.63,0.69}{\textit{#1}}}
\newcommand{\CommentVarTok}[1]{\textcolor[rgb]{0.38,0.63,0.69}{\textbf{\textit{#1}}}}
\newcommand{\ConstantTok}[1]{\textcolor[rgb]{0.53,0.00,0.00}{#1}}
\newcommand{\ControlFlowTok}[1]{\textcolor[rgb]{0.00,0.44,0.13}{\textbf{#1}}}
\newcommand{\DataTypeTok}[1]{\textcolor[rgb]{0.56,0.13,0.00}{#1}}
\newcommand{\DecValTok}[1]{\textcolor[rgb]{0.25,0.63,0.44}{#1}}
\newcommand{\DocumentationTok}[1]{\textcolor[rgb]{0.73,0.13,0.13}{\textit{#1}}}
\newcommand{\ErrorTok}[1]{\textcolor[rgb]{1.00,0.00,0.00}{\textbf{#1}}}
\newcommand{\ExtensionTok}[1]{#1}
\newcommand{\FloatTok}[1]{\textcolor[rgb]{0.25,0.63,0.44}{#1}}
\newcommand{\FunctionTok}[1]{\textcolor[rgb]{0.02,0.16,0.49}{#1}}
\newcommand{\ImportTok}[1]{#1}
\newcommand{\InformationTok}[1]{\textcolor[rgb]{0.38,0.63,0.69}{\textbf{\textit{#1}}}}
\newcommand{\KeywordTok}[1]{\textcolor[rgb]{0.00,0.44,0.13}{\textbf{#1}}}
\newcommand{\NormalTok}[1]{#1}
\newcommand{\OperatorTok}[1]{\textcolor[rgb]{0.40,0.40,0.40}{#1}}
\newcommand{\OtherTok}[1]{\textcolor[rgb]{0.00,0.44,0.13}{#1}}
\newcommand{\PreprocessorTok}[1]{\textcolor[rgb]{0.74,0.48,0.00}{#1}}
\newcommand{\RegionMarkerTok}[1]{#1}
\newcommand{\SpecialCharTok}[1]{\textcolor[rgb]{0.25,0.44,0.63}{#1}}
\newcommand{\SpecialStringTok}[1]{\textcolor[rgb]{0.73,0.40,0.53}{#1}}
\newcommand{\StringTok}[1]{\textcolor[rgb]{0.25,0.44,0.63}{#1}}
\newcommand{\VariableTok}[1]{\textcolor[rgb]{0.10,0.09,0.49}{#1}}
\newcommand{\VerbatimStringTok}[1]{\textcolor[rgb]{0.25,0.44,0.63}{#1}}
\newcommand{\WarningTok}[1]{\textcolor[rgb]{0.38,0.63,0.69}{\textbf{\textit{#1}}}}
\usepackage{graphicx,grffile}
\makeatletter
\def\maxwidth{\ifdim\Gin@nat@width>\linewidth\linewidth\else\Gin@nat@width\fi}
\def\maxheight{\ifdim\Gin@nat@height>\textheight\textheight\else\Gin@nat@height\fi}
\makeatother
% Scale images if necessary, so that they will not overflow the page
% margins by default, and it is still possible to overwrite the defaults
% using explicit options in \includegraphics[width, height, ...]{}
\setkeys{Gin}{width=\maxwidth,height=\maxheight,keepaspectratio}
\setlength{\emergencystretch}{3em}  % prevent overfull lines
\providecommand{\tightlist}{%
  \setlength{\itemsep}{0pt}\setlength{\parskip}{0pt}}
\setcounter{secnumdepth}{0}

% set default figure placement to htbp
\makeatletter
\def\fps@figure{htbp}
\makeatother

% Copyright 2004 by Madhusudan Singh <madhusudan.singh@gmail.com>
%
% This file may be distributed and/or modified
%
% 1. under the LaTeX Project Public License and/or
% 2. under the GNU Public License.
%
% See the file doc/licenses/LICENSE for more details.

\mode<presentation>

\RequirePackage[utf8]{inputenc}
\RequirePackage{lmodern}
\RequirePackage[T1]{fontenc}
\RequirePackage{listings}
\RequirePackage{colortbl}
\RequirePackage{keystroke}

\definecolor{olive}{HTML}{80C040}
\definecolor{purple}{HTML}{8000E0}
\definecolor{darkblue}{HTML}{000080}
\definecolor{darkgrey}{HTML}{808080}
\definecolor{middlegrey}{HTML}{C0C0C0}
\definecolor{lightgrey}{HTML}{F0F0F0}
\definecolor{lightblue}{HTML}{01A0DE}


\setbeamercolor*{palette primary}{fg=white,bg=white}
\setbeamercolor*{palette quarternary}{bg=white}
\setbeamercolor*{normal text}{fg=black}
\setbeamercolor*{titlelike}{fg=black}
\setbeamercolor*{example text}{use=palette secondary} %,palette secondary.bg
%\setbeamercolor*{sidebar}{fg=yellow,bg=black}
\setbeamercolor*{separation line}{fg=black}
\setbeamercolor*{structur}{fg=yellow,bg=yellow}
\setbeamercolor*{alerted text}{fg=red}
%\setbeamercolor*{section in toc}{normal text}

%Variante 1: Grau in grau
 \setbeamercolor*{block body}{fg=black,bg=lightgrey}
 \setbeamercolor*{block title}{fg=black,bg=middlegrey}
 \setbeamercolor*{item}{fg=middlegrey}
\setbeamercolor*{section in toc}{fg=black}
 \setbeamercolor*{palette secondary}{bg=lightgrey}
 \setbeamercolor*{palette tertiary}{bg=middlegrey}
\setbeamercolor{section number projected}{bg=black}
%Variante 2: Schwarz und rot
%\setbeamercolor*{block title}{fg=white,bg=black}
%\setbeamercolor*{block body}{fg=black,bg=lightgrey}
%\setbeamercolor*{item}{fg=red}
%\setbeamercolor*{palette secondary}{bg=black,fg=white}
%\setbeamercolor*{palette tertiary}{bg=lightgrey}




\newcommand{\rhbinsertclassification}{OFFEN}
\newcommand{\rhbClassification}[1]{\renewcommand{\rhbinsertclassification}{#1}}
\newcommand{\rhbTH}[1]{ \cellcolor{black} {\color{white}#1} }

\newenvironment{rhbTable}[1]
	{ \begin{center} \begin{tabular}{#1}\hline}
	{\hline \end{tabular} \end{center}}
\newcommand{\rhbcolorrow}{\rowcolor{lightgrey}}

\newcommand{\rhbinsertlogo}{\includegraphics[height=1cm]{logo}}
\newcommand{\rhbColumns}[2]
	{
		\begin{columns}[T]
		\begin{columns}{.5\textwidth}
		#1
		\end{column}
		\begin{column}{.5\textwidth}
		#2
		\end{column}
		\end{columns}
	
	}


\useinnertheme{rectangles}
\useoutertheme{infolines}



% modified templates

\addtobeamertemplate{title page}{\centering \includegraphics[height=2cm]{logo}}{}


\setbeamertemplate{frametitle}{%
\begin{minipage}[b][1.2cm][c]{0.9\textwidth} \rhbinsertlogo%
\hspace{0.5cm} \begin{beamercolorbox}[left]{white}%

	
\ifx\insertframesubtitle\@empty%
\raisebox{8pt}{\insertframetitle}%
\else%
{{\insertframetitle}\par}
{\usebeamerfont{subtitle}\usebeamercolor[fg]{framesubtitle}\insertframesubtitle\par}
\fi
\end{beamercolorbox}
\end{minipage}%
}


\setbeamertemplate{headline}
{
  \leavevmode%
  \hbox{%
  \begin{beamercolorbox}[wd=.5\paperwidth,ht=2.65ex,dp=1.5ex,center]{section in head/foot}%
    \usebeamerfont{section in head/foot}\insertsectionhead\hspace*{2ex}
  \end{beamercolorbox}%
  \begin{beamercolorbox}[wd=.5\paperwidth,ht=2.65ex,dp=1.5ex,center]{subsection in head/foot}%
    \usebeamerfont{subsection in head/foot}\hspace*{2ex}\insertsubsectionhead
  \end{beamercolorbox}}%
  \vskip0pt%
}




%\makeatother
%\setbeamertemplate{headline}%
%{
%\ifnum \insertframenumber=1
%\leavevmode%
%\hbox{%
%\begin{beamercolorbox}[wd=\paperwidth,ht=2.25ex,dp=1ex,center]{palette tertiary}%
% \usebeamerfont{author in head/foot}\rhbinsertclassification%
%\end{beamercolorbox}}%
%\vskip0pt%
%\else
%\leavevmode%
%\hbox{%
%\begin{beamercolorbox}[wd=\paperwidth,ht=2.25ex,dp=1ex,left]{palette tertiary}%
% \usebeamerfont{author in head/foot}\center\rhbinsertclassification%
%\end{beamercolorbox}}%
%\vskip0pt%
%\fi
%}

\makeatletter



\setbeamerfont{block title}{size={}}
%\setbeamercolor{titlelike}{parent=structure,bg=white}
\mode<all>

\title{Example Presentation}
\providecommand{\subtitle}[1]{}
\subtitle{Pentesting Department}
\author{David Lassig}
\date{10.August 2018}

\begin{document}
\frame{\titlepage}

\hypertarget{idee}{%
\section{Idee}\label{idee}}

\begin{frame}{Initiator}
\protect\hypertarget{initiator}{}

\includegraphics[width=3.125in,height=2.08333in]{images/lorawan_news.png}

\end{frame}

\begin{frame}{}
\protect\hypertarget{section}{}

\begin{figure}
\hypertarget{id}{%
\centering
\includegraphics[width=3.125in,height=2.08333in]{images/lorawan_usecases.gif}
\caption{Anwendungsmöglichkeiten für LoRaWAN}\label{id}
}
\end{figure}

\end{frame}

\begin{frame}{}
\protect\hypertarget{section-1}{}

\begin{figure}
\hypertarget{id}{%
\centering
\includegraphics[width=3.125in,height=2.08333in]{images/lorawan_coverage.png}
\caption{Beispiel Reichweite einzelner Gateway Antenne}\label{id}
}
\end{figure}

\end{frame}

\hypertarget{lorawan101}{%
\section{LoRaWan101}\label{lorawan101}}

\begin{frame}{Was ist LORA?}
\protect\hypertarget{was-ist-lora}{}

Funkgebundene Modulation zur Kommunikationsverbindung über Langstrecke
in LPWAN\footnote<.->{Low-Power, Wide-Area-Network} (Layer 1) zur
Optimierung von

\begin{itemize}
\tightlist
\item
  Batterielebensdauer
\item
  Robustheit
\item
  Entfernung
\item
  Kosten
\end{itemize}

\end{frame}

\begin{frame}{Modulation}
\protect\hypertarget{modulation}{}

\begin{itemize}
\tightlist
\item
  LoRa benutzt CSS für robuste Übertragung.
\end{itemize}

\begin{figure}
\hypertarget{id}{%
\centering
\includegraphics[width=3.125in,height=2.08333in]{images/css_example.png}
\caption{Beispiel eines Chirps}\label{id}
}
\end{figure}

\end{frame}

\hypertarget{umsetzung}{%
\section{Umsetzung}\label{umsetzung}}

\begin{frame}{Architecture}
\protect\hypertarget{architecture}{}

\includegraphics[width=4.16667in,height=3.64583in]{images/sequence.pdf}

\end{frame}

\begin{frame}[fragile]{Set up RX}
\protect\hypertarget{set-up-rx}{}

\begin{Shaded}
\begin{Highlighting}[]
    \KeywordTok{def}\NormalTok{ set_up_rx(}\VariableTok{self}\NormalTok{):}

        \ControlFlowTok{if} \KeywordTok{not} \VariableTok{self}\NormalTok{.create_lora():}
            \VariableTok{self}\NormalTok{.pdeb(}\StringTok{"Lora Socket failed"}\NormalTok{)}
        \ControlFlowTok{else}\NormalTok{:}
            \VariableTok{self}\NormalTok{.pdeb(}\StringTok{"Lora Socket Raw RX succesfully created"}\NormalTok{)}
            \ControlFlowTok{if} \VariableTok{self}\NormalTok{.basemode }\OperatorTok{==} \StringTok{"master"}\NormalTok{:}
                \VariableTok{self}\NormalTok{.s.settimeout(}\DecValTok{10}\NormalTok{)}
\NormalTok{        time.sleep(}\DecValTok{1}\NormalTok{)}
\end{Highlighting}
\end{Shaded}

\end{frame}

\begin{frame}[fragile]{Receive Mode}
\protect\hypertarget{receive-mode}{}

\begin{Shaded}
\begin{Highlighting}[]

    \KeywordTok{def}\NormalTok{ receive_mode(}\VariableTok{self}\NormalTok{):}
        \ControlFlowTok{try}\NormalTok{:}
\NormalTok{            msg }\OperatorTok{=} \StringTok{""}
            \VariableTok{self}\NormalTok{.set_up_rx()}
            \ControlFlowTok{while}\NormalTok{(}\VariableTok{True}\NormalTok{):}
\NormalTok{                result }\OperatorTok{=} \VariableTok{self}\NormalTok{.receive()}
                \ControlFlowTok{if}\NormalTok{ result }\OperatorTok{==} \DecValTok{2}\NormalTok{: }\CommentTok{# receive part}
\NormalTok{                    msg }\OperatorTok{+=} \VariableTok{self}\NormalTok{.msg}
                    \VariableTok{self}\NormalTok{.send(}\StringTok{"ack"}\NormalTok{,}\VariableTok{self}\NormalTok{.cb_output_payload)}
\NormalTok{                    time.sleep(}\DecValTok{1}\NormalTok{)}
                \ControlFlowTok{if}\NormalTok{ result }\OperatorTok{==} \DecValTok{1}\NormalTok{: }\CommentTok{#receive finish}
\NormalTok{                    msg }\OperatorTok{+=} \VariableTok{self}\NormalTok{.msg}
                    \VariableTok{self}\NormalTok{.out_msg(msg,}\VariableTok{self}\NormalTok{.last_device_id)}
                    \VariableTok{self}\NormalTok{.send(}\StringTok{"ack"}\NormalTok{,}\VariableTok{self}\NormalTok{.cb_output_payload)}
\NormalTok{                    time.sleep(}\DecValTok{1}\NormalTok{)}
                    \VariableTok{self}\NormalTok{.mode }\OperatorTok{=} \StringTok{"tx"}
\NormalTok{                    time.sleep(}\DecValTok{1}\NormalTok{)}
                    \ControlFlowTok{break}
                \ControlFlowTok{if}\NormalTok{ result }\OperatorTok{==} \DecValTok{0}\NormalTok{: }\CommentTok{# wrong device_id}
                    \VariableTok{self}\NormalTok{.pdeb(}\StringTok{"Not our cup of tea"}\NormalTok{)}
\NormalTok{                    time.sleep(}\DecValTok{1}\NormalTok{)}
                    \ControlFlowTok{break}
        \ControlFlowTok{except}\NormalTok{ socket.timeout:}
            \VariableTok{self}\NormalTok{.pdeb(}\StringTok{"Master got Receiver Timeout, changing back to TX"}\NormalTok{)}
            \VariableTok{self}\NormalTok{.mode }\OperatorTok{=} \StringTok{"tx"}
\end{Highlighting}
\end{Shaded}

\end{frame}

\hypertarget{ausblick}{%
\section{Ausblick}\label{ausblick}}

\begin{frame}{ToDo}
\protect\hypertarget{todo}{}

\begin{itemize}
\tightlist
\item
  ``Protokollhärtung'' - Wiederholtes Senden von Paketen
\item
  Datenhaltung und Auswertung im Webserver
\item
  Wechsel auf andere Hardwareplattform ( Dragino )
\item
  Wechsel auf andere Modulation ???
\end{itemize}

\end{frame}

\end{document}
